%!TEX root = Main.tex
\documentclass[Main]{subfiles}

\begin{document}
\section{CO$_2$ emission data} % (fold)
\label{sub:c02_emission_data}

	Data on CO$_2$ emission from electricity production to the Danish power grid is provided by the company Energinet.dk that oversees power distribution on the entire Danish power grid.

	Two types of data are provided:
	\begin{itemize}
		\item Prognosis \cite{EnerginetFTPPrognosis:Online}
		\item Online data \cite{EnerginetFTPOnline:Online}
	\end{itemize}

	All data is given in amount of CO$_2$ per unit of energy, in this case: 
	$\sfrac{g}{kWh}$.
	This format is a very good representation for our use case.
	That is because it gives a good holistic view of how big a fraction of the energy is from sustainable or \emph{green} sources.
	
	This is in contrast to e.g. a measure of CO$_2$ emission per hour.
	Such a measure will be greatly influenced by total energy production/consumption, without regard for situations where there for example is a very low consumption of power, but it is almost exclusively coming from highly polluting coal powered plants.

	\subsection{Prognosis data} % (fold)
	\label{sub:prognosis}
		Prognosis data is a prediction of how much power is going to be consumed over the coming 24 hours in 1 hour increments.
		The prognosis is released every day at 15:00 when all energy producers report in their expected production and demand.
		These expectations are based on complex models taking into account factors like history, weather forecast and season.
		After its release the prognosis can be gradually adjusted at 15 min. intervals if supply/demand changes, influenced by e.g. changes in the weather forecast, abnormal consumption or fluctuating prices at the Scandinavian energy exchange, Nord Pool Spot \cite{NordPool:Online}.

		With data about the individual sources of power (large regional power plants, biomass burning municipal power plants, wind/solar, etc.) an estimate of the average CO$_2$ emission per kWh is then produced.

		% subsection prognosis (end)

	\subsection{Online data} % (fold)
	\label{sub:online_data}
		Contrary to the prognosis data described in Section \ref{sub:prognosis} the online data is real-time data on actual production of electricity and hence CO$_2$ emission.
		The online data is updated at 5 min. intervals during all 24 hours of the day.

		This, and the fact that the coherence time i.e. the time it takes for the data to change significantly, is considerably higher than 5 min., removes a lot of the uncertainty, resident in the prognosis data.

		% subsection online_data (end)

	% section c02_emission_data (end)



\end{document}
